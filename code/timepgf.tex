\documentclass[11pt,leqno]{article}
\usepackage{amsfonts, amsmath}
\usepackage{fancyhdr}
\usepackage[
  letterpaper=true,
  colorlinks=true,
  linkcolor=red,
  citecolor=red,
  pdfpagemode=None]{hyperref}

\usepackage{graphicx}
\usepackage{color}
\usepackage{tikz}
\usepackage{pgfplots}
\usepackage{listings}

\title{Performance of basic matmul}
\author{David Bindel}

\begin{document}

\pagestyle{fancy}
\lhead{Bindel, Fall 2011}
\rhead{Applications of Parallel Computers (CS 5220)}
\fancyfoot{}

\maketitle

This is a simple \LaTeX\ file to demonstrate how I use the
{\tt listings} package to include nicely typeset code in my documents,
and how I use {\tt pgfplots} to generate nice-looking plots.
In order to compile this document, you will need to have a \LaTeX\ 
installation with PGF and TikZ installed.

The code is shown in Figure~\ref{fig-naive-matmul}, and the performance
is shown in Figure~\ref{fig-naive-perf}.  The performance is never that
high, but it falls off a clif for the dimensions greater than 400.
Notice the interesting wiggles in the performance curve due to cache
associativity effects.

\begin{figure}
\lstset{language=c,frame=lines,columns=flexible}
\lstinputlisting{basic_dgemm.c}
\caption{Naive matrix-matrix multiplication code in C.}
\label{fig-naive-matmul}
\end{figure}

\begin{figure}
\begin{tikzpicture}
\begin{axis}[
  /pgf/number format/set thousands separator={},
  width=0.8\textwidth, height=3in,
  xlabel={$n$},
  ylabel={MFlop/s}]
\addplot file {timing-basic.dat};
\end{axis}
\end{tikzpicture}
\caption{Performance of naive matrix-matrix multiplication code.}
\label{fig-naive-perf}
\end{figure}

\end{document}
